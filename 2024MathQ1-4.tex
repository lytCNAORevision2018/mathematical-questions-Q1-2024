\documentclass[a4paper,12pt]{article}
\usepackage{ctex}
\usepackage{amsmath,amsbsy,amsfonts}
\usepackage{geometry}
\geometry{left=2.0cm,right=2.0cm,top=2.5cm,bottom=2.5cm}
\title{2024年寒假特制试卷:丁卷}
\author{一只波兰立陶宛联邦球}
\date{2024年2月17日}
\begin{document}
\maketitle
\section*{注意事项}
本试卷共15题,每题10分,卷面总分150分。考试时间4小时。
范围:高等数学。
\subsection*{1}\noindent 设实数$\alpha>0$,数列$\left\{u_{n}\right\}(n\geq 1)$满足条件:
\begin{equation*}
	u_{1}>0\quad u_{n+1}=u_{n}+\frac{1}{n^{\alpha}u_{n}}
\end{equation*}
\noindent $\left(1\right)$证明:当且仅当$\alpha>1$时该数列收敛。\\
\noindent$\left(2\right)$若$\alpha>1$,记
\begin{equation*}
	\lambda := \lim\limits_{n\rightarrow\infty}u_{n}
\end{equation*}
求证:
\begin{equation*}
	\lambda-u_{n}\sim\frac{1}{\lambda\left(\alpha-1\right)n^{\alpha-1}}\quad\left(n\rightarrow\infty\right)
\end{equation*}
\noindent$\left(3\right)$对于充分大的$n$,求证:\\如果$\alpha=1$,有\begin{equation*}
	u_{n}\sim\sqrt{2\log n}\quad\left(n\rightarrow\infty\right)
\end{equation*}
如果$0<\alpha<1$,有:
\begin{equation*}
	u_{n}\sim\sqrt{\frac{2}{1-\alpha}}n^{\frac{1-\alpha}{2}}\quad\left(n\rightarrow\infty\right)
\end{equation*}
\subsection*{2}\noindent
设参变量$\alpha=\alpha\left(x,y\right)$的一阶偏导数存在。函数$z=z\left(x,y\right)$由下列方程确定:
\begin{equation*}
	z=\alpha x+2\left(\alpha^2+1\right)y+2\alpha^2\quad x+4\alpha y+4\alpha=0
\end{equation*}
求证:
\begin{equation*}
	\frac{\partial^2 z}{\partial x^2}+\frac{\partial^2 z}{\partial y^2}=\left(\frac{\partial^2 z}{\partial x\partial y}\right)^2
\end{equation*}

\subsection*{3}\noindent 设函数$f\in C^{\left[0,1\right]}$,$f\left(0\right)=0$。证明:对于任意$n\in\mathbb{N}$,有:
\begin{equation*}
	\int_{0}^{1}\left|f\left(x\right)\right|^{n}\left|f^{\prime}\left(x\right)\right|\mathrm{d}x\leq \frac{1}{n+1}\int_{0}^{1}\left|f^{\prime}\left(x\right)\right|^{n+1}\mathrm{d}x
\end{equation*}
等号成立的充分必要条件为$f$仅为齐次线性函数$\left(f\left(x\right)=ax\right)$。
\subsection*{4}\noindent 设函数$f\in C^{1}\left[0,\infty\right)$,当$x\rightarrow 0$时$f\left(x\right)\rightarrow0$,并且对于某个实数$a>-1$,积分\begin{equation*}
	\int_{0}^{\infty}t^{a+1}f^{\prime}\left(x\right)\mathrm{d}t
\end{equation*}
绝对收敛。证明:\\
\noindent$\left(1\right)$$\lim\limits_{x\rightarrow\infty}x^{a+1}f\left(x\right)=0$\\
\noindent$\left(2\right)$积分$\int_{0}^{\infty}t^{a}f\left(t\right)\mathrm{d}t$收敛,其值为:
\begin{equation*}
	\frac{1}{a+1}\int_{0}^{\infty}t^{a+1}f^{\prime}\left(t\right)\mathrm{d}t
\end{equation*}
\subsection*{5}\noindent$\left(1\right)$设$\alpha,\beta >0$。证明:
\begin{equation*}
	I\left(\alpha,\beta\right)=\int_{0}^{\frac{\pi}{2}}\log\left(\alpha\cos^2\theta+\beta\sin^2\theta\right)\mathrm{d}\theta=\pi\log\left(\frac{\sqrt{\alpha}+\sqrt{\beta}}{2}\right)
\end{equation*}
\noindent $\left(2\right)$设$\alpha>1$。证明:
\begin{equation*}
	J\left(\alpha\right)=\int_{0}^{\frac{\pi}{2}}\log\left(\alpha^2-\sin^2\phi\right)\mathrm{d}\phi=\pi\log\left(\frac{\alpha+\sqrt{\alpha^2-1}}{2}\right)
\end{equation*}
\noindent $\left(3\right)$证明以上的两个计算公式是等价的。$\left(\alpha\neq\beta\right)$。
\subsection*{6}\noindent$\left(1\right)$设$\alpha<\beta$。求下列曲面围成的立体体积。\begin{equation*}\begin{aligned}
	&x^2+y^2+z^2=2az\\
	&x^2+y^2=z^2\tan^2\alpha\\
	&x^2+y^2=z^2\tan^2\beta\\	
	\end{aligned}
\end{equation*}
\noindent$\left(2\right)$求曲面
\begin{equation*}
	\left(\frac{x^2}{a^2}+\frac{y^2}{b^2}+\frac{z^2}{c^2}\right)^{n}=z^{2n-1}\quad\left(n\in\mathbb{N};a,b,c>0\right)
\end{equation*}
围成的立体体积。
\subsection*{7}\noindent$\left(1\right)$计算:
\begin{equation*}
	\int_{0}^{\frac{\pi}{2}}\log\left(\frac{a+b\sin\theta}{a-b\sin\theta}\right)\frac{\mathrm{d}\theta}{\sin\theta}
\end{equation*}
\noindent$\left(2\right)$证明:
\begin{equation*}
	\int_{\frac{\pi}{2}-\alpha}^{\frac{\pi}{2}}\sin\theta\arccos\left(\frac{\cos\alpha}{\sin\theta}\right)\mathrm{d}\theta=\frac{\pi}{2}\left(1-\cos\alpha\right)\quad\left(0\leq\alpha\leq\frac{\pi}{2}\right)
\end{equation*}
\subsection*{8}\noindent 求函数
$f\left(x,y,z\right)=x^2+y^2+\left(z-1\right)^2$在约束条件$3x^2-2xy+2y^2-2x-6y+7=0$下的最值。
\subsection*{9}\noindent$\left(1\right)$使用Green公式计算:
\begin{equation*}
	I_{1}=\int_{C}\frac{e^x\left(x\sin y-y\cos y \right)\mathrm{d}x+e^{x}\left(x\cos y+y\sin y\right)\mathrm{d}y}{x^2+y^2}
\end{equation*}
\noindent$\left(2\right)$计算积分:
\begin{equation*}
	I_{2}=\iint_{\Sigma}\left(x-y-z\right)\mathrm{d}y\mathrm{d}z+\left[2y+\sin\left(z+x\right)\right]\mathrm{d}z\mathrm{d}x+\left[3z+\exp\left(x+y\right)\right]\mathrm{d}x\mathrm{d}y
\end{equation*}
其中$\Sigma:\left|x-y+z\right|+\left|y-z+x\right|+\left|z-x+y\right|=1$
\subsection*{10}\noindent$\left(1\right)$设
\begin{equation*}
	f\left(x\right)=\left\{
	\begin{aligned}
	\frac{\left(\pi-1\right)x}{2}&,0\leq x\leq 1\\
	\frac{\pi-x}{2}&,1<x<\pi
	\end{aligned}
	\right.
\end{equation*}
并且当$-\pi<x<0$时取$f(x)=-f(x)$。求$f(x)$在$\left(-\pi,\pi\right)$上的Fourier展开。\\
\noindent$\left(2\right)$求无穷级数\begin{equation*}
	\sum_{n=1}^{\infty}\frac{\sin^2 n}{n^4}
\end{equation*}
的和。
\subsection*{11}\noindent 求曲线$x^3-3xy+y^3=0$自身围成的区域面积,以及它与其渐近线$x+y+a=0$之间的面积。
\subsection*{12}\noindent 设函数$u\left(x,y\right)$在$\Omega=[0,1]\times\left[0,1\right]$上连续,并且有:
\begin{equation*}
	\iint_{\Omega}\left|\frac{\partial u\left(x,y\right)}{\partial y}\right|^{2}\mathrm{d}x\mathrm{d}y<\infty
\end{equation*}
求证:
\begin{equation*}
	\int_{0}^{1}\left|u\left(x,0\right)\right|^2\mathrm{d}x\leq\sqrt{5}\left(\iint_{\Omega}\left|u\left(x,y\right)\right|^2\mathrm{d}x\mathrm{d}y\right)^{\frac{1}{2}}\left(
	\iint_{\Omega}\left|u\left(x,y\right)\right|^2\mathrm{d}x\mathrm{d}y+\iint_{\Omega}\left|\frac{\partial u\left(x,y\right)}{\partial y}\right|^2\mathrm{d}x\mathrm{d}y\right)^{\frac{1}{2}}
\end{equation*}
\subsection*{13}\noindent 设函数
$f\in C^2\left(0,\infty\right)$,$f\left(x\right)\rightarrow0\left(x\rightarrow\infty\right)$。对于某个$\lambda$,$f^{\prime\prime}\left(x\right)+\lambda f^{\prime}\left(x\right)$
有界。求证:$f^{\prime}\left(x\right)\rightarrow0\left(x\rightarrow\infty\right)$
\subsection*{14}\noindent 设实值函数$f\left(x\right)$及其一阶导数在区间$\left[a,b\right]$上均连续,且$f\left(a\right)=0$。求证:\\
\noindent$\left(1\right)$
\begin{equation*}
	\max\limits_{a\leq x\leq b}\left|f\left(x\right)\right|\leq\sqrt{b-a}\left(\int_{a}^{b}\left|f^{\prime}\left(x\right)\right|^2\mathrm{d}x\right)^{\frac{1}{2}}
\end{equation*}
\noindent$\left(2\right)$\begin{equation*}
	\int_{a}^{b}f^{2}\left(x\right)\mathrm{d}x\leq\frac{1}{2}\left(b-a\right)^2\int_{a}^{b}\left|f^{\prime}\left(x\right)\right|^2\mathrm{d}x
\end{equation*}
\subsection*{15}\noindent 设$u_{0}=1$,定义:
\begin{equation*}
	u_{n}=\int_{0}^{1}\prod_{k=0}^{n-1}\left(t-k\right)\mathrm{d}t\quad\left(n\geq 1\right)
\end{equation*}
证明在$\left[-1,1\right]$上,级数
\begin{equation*}
	\sum_{n=1}^{\infty}\frac{u_{n}x^{n}}{n!}
\end{equation*}
收敛,并求其和。
\end{document}
