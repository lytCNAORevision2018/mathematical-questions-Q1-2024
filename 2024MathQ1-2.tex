\documentclass[a4paper,12pt]{article}
\usepackage{ctex}
\usepackage{amsmath,amsbsy,amsfonts}
\usepackage{geometry}
\geometry{left=2.0cm,right=2.0cm,top=2.5cm,bottom=2.5cm}


\title{2024年寒假特制试卷:乙卷}
\author{一只波兰立陶宛联邦球}
\date{2024年2月6日}

\begin{document}

\maketitle
\section*{注意事项}
本试卷共15题,每题10分,卷面总分150分。考试时间4小时。

范围:高等数学。
\subsection*{1}\noindent 已知$0<x_{1}<1$,$x_{n+1}=x_{n}\left(1-x_{n}\right)$,$n=1,2,\dots$。\\
\noindent $\left(1\right)$证明:$\left\{x_{n}\right\}$收敛,并求其极限。\\
\noindent $\left(2\right)$证明:\begin{equation*}
	\lim\limits_{n\rightarrow\infty}nx_{n}=1
\end{equation*}
\noindent$\left(3\right)$
假设$0<p<1$,$0<x_{1}<\frac{1}{p}$,其中$0<p\leq 1$,同时
$x_{n+1}=x_{n}\left(1-px_{n}\right), n\in\mathbb{N}_{+}$。

求证:
\begin{equation*}
	\lim\limits_{n\rightarrow\infty}nx_{n}=\frac{1}{p}
\end{equation*}
\subsection*{2}
\paragraph{$\left(1\right)$}\noindent 求
\begin{equation*}
	f(x)=\frac{\mathrm{d}}{\mathrm{d}x}\left(\frac{\cos x-1}{x}\right)
\end{equation*}的麦克劳林展式,并求级数
\begin{equation*}
\sum_{n=1}^{\infty}\left(-1\right)^n\frac{2n-1}{2n!}\left(\frac{\pi}{2}\right)^{2n}
\end{equation*}
的和。
\paragraph{$\left(2\right)$}\noindent 请将级数\begin{equation*}
	\sum_{n=1}^{\infty}\frac{\left(-1\right)^{n-1}}{2^{n-1}}\frac{x^{2n-1}}{\left(2n-1\right)!}
\end{equation*}的和函数展开成$\left(x-1\right)$的幂级数。
\subsection*{3}
\paragraph{$\left(1\right)$}\noindent 计算三重积分:
\begin{equation*}
	I=\iiint_{V}x\cdot\exp(y+z)\mathrm{d}x\mathrm{d}y\mathrm{d}z
\end{equation*}
其中,$V$是曲面$y=x^2$,$y=x$,$x+y+z=2$和$z=0$围城的区域。
\paragraph{$\left(2\right)$}\noindent 设$f(x)$在$t\neq 0$时一阶连续可导,且$f(1)=0$。求函数$f(x^2+y^2)$,使曲线积分:
\begin{equation*}
	I=\int_{L}y\left[2-f\left(x^2+y^2\right)\right]\mathrm{d}x+xf\left(x^2+y^2\right)\mathrm{d}y
\end{equation*}
与路径无关。其中$L$是不经过原点的光滑曲线。
\subsection*{4}
\noindent \paragraph{$\left(1\right)$}
已知:
\begin{equation*}
	\int_{0}^{\infty}\frac{\sin x}{x}\mathrm{d}x=\frac{\pi}{2}
\end{equation*}
试计算
\begin{equation*}
	\int_{0}^{\infty}\int_{0}^{\infty}\frac{\sin x\sin\left(x+y\right)}{x\left(x+y\right)}\mathrm{d}x\mathrm{d}y
\end{equation*}
\noindent\paragraph{$\left(2\right)$}试求极限:
\begin{equation*}
	\lim\limits_{n\rightarrow\infty}\sqrt{n}\left(1-\sum_{k=1}^{n}\frac{1}{n+\sqrt{k}}\right)
\end{equation*}
\noindent\paragraph{$\left(3\right)$}试计算:
\begin{equation*}
	T:=\int_{0}^{2\pi}\mathrm{d}\phi\int_{0}^{\pi}\exp\left[\sin\theta\left(\cos\phi-\sin\phi\right)\right]\sin\theta\mathrm{d}\theta
\end{equation*}
\noindent\paragraph{$\left(4\right)$}计算积分:
\begin{equation*}
	\int_{0}^{\pi}\ln\left(2+\cos x\right)\mathrm{d}x
\end{equation*}
\subsection*{5}\noindent 设:
\begin{equation*}
	u_{n}\left(x\right)=\exp\left(-nx\right)+\frac{1}{n\left(n+1\right)}x^{n+1},n=1,2,\dots
\end{equation*}求函数项级数$\sum_{n=1}^{\infty}u_{n}\left(x\right)$的收敛域与和函数。
\subsection*{6}\noindent$\left(1\right)$设$f(x)$在
$\left[a,b\right]$上连续,$\left(a,b\right)$内可导。$1<a<b$,$f^{\prime}\left(x\right)\neq 0$。求证:存在$\xi,\eta\in\left(a,b\right)$,使得:
\begin{equation*}
	\frac{f^{\prime}\left(\xi\right)}{f^{\prime}\left(\eta\right)}=\frac{1}{1+\ln\eta}\frac{b\ln b-a\ln a}{b-a}
\end{equation*}

\noindent$\left(2\right)$
设$f(x)$在
$\left[a,b\right]$上连续,$\left(a,b\right)$内可导,$0<a<b$。求证:存在$\gamma\in\left(a,b\right)$使得:
\begin{equation*}
	2\gamma\left[f(b)-f(a)\right]=\left(b^2-a^2\right)f\left(\gamma\right)
\end{equation*}
\noindent$\left(3\right)$设$f(x)$在
$\left[0,1\right]$上连续,$\left(0,1\right)$内可导。$f(0)=0$,$f(1)=1$。求证:对于任何正数$a,b$,在$\left(0,1\right)$内存在不相等的两个数$m$和$n$使得:
\begin{equation*}
	\frac{a}{f^{\prime}\left(m\right)}+\frac{b}{f^{\prime}\left(n\right)}=a+b
\end{equation*}
\noindent$\left(4\right)$设$f(x)$在
$\left[a,b\right]$上连续,$\left(a,b\right)$内可导。其中,$a>0$。求证:存在$\alpha,\beta\in\left(a,b\right)$,使得:
\begin{equation*}
	f^{\prime}\left(\alpha\right)=\frac{a+b}{2\beta}f^{\prime}\left(\beta\right)
\end{equation*}
\subsection*{7}计算以下两个积分:\noindent\paragraph{$\left(1\right)$}\begin{equation*}
	\int_{0}^{+\infty}\sin(x^2)\mathrm{d}x
\end{equation*}
\noindent\paragraph{$\left(2\right)$}\begin{equation*}
	\int_{0}^{+\infty}\cos(x^2)\mathrm{d}x
\end{equation*}
\subsection*{8}
\noindent\paragraph{$\left(1\right)$}设函数$f(x)$,$g(x)$在闭区间$\left[a,b\right]$上连续,且$f(x)$严格单调递增,同时函数$g(x)$满足$0<g(x)<1$。证明:
\begin{equation*}
	\int_{a}^{a+\int_{a}^{b}g(t)\mathrm{d}t}f(x)\mathrm{d}x\leq\int_{a}^{b}f(x)g(x)\mathrm{d}x
\end{equation*}

\noindent\paragraph{$\left(2\right)$}
讨论\begin{equation*}
	F\left(x\right)=-\frac{1}{2}\left(1+\left(\frac{1}{e}\right)\right)+\int_{-1}^{1}|x-t|\exp(-t^2)\mathrm{d}t=0
\end{equation*}
在闭区间$\left[-1,1\right]$上实数根的个数。
\subsection*{9}\noindent 请使用泰勒公式说明函数在某区间上的凹凸性和在该区间上二阶导数之间的关系。已知该函数在定义域中给定任意区间上的二阶导数都存在。
\subsection*{10}
\noindent 设$f(x)$在$x>1$时具有连续的二阶导数,$f(1)=0$,$f^{\prime}(1)=1$。$z=\left(x^2+y^2\right)f\left(x^2+y^2\right)$满足
\begin{equation*}
	\frac{\partial^2 z}{\partial x^2}+\frac{\partial^2 z}{\partial y^2}=0
\end{equation*}
求$f(x)$在$x>1$时的最值。
\subsection*{11}\noindent 已知三元函数$f\left(x,y,z\right)$在$\mathbb{R}^3$上具有连续的二阶偏导数。假设$\mathbb{R}^3$中光滑简单封闭曲面的全体为$\Sigma$,对于$S\in \Sigma$,用$\vec{n}$表示$S$的外法线单位方向向量,$\frac{\partial f}{\partial \vec{n}}$表示$f$沿$\vec{n}$的方向导数。
\noindent $\left(1\right)$证明:$f$在$\mathbb{R}^3$上满足\begin{equation*}
	\frac{\partial^2 f}{\partial x^2}+	\frac{\partial^2 f}{\partial y^2}+	\frac{\partial^2 f}{\partial z^2}=0
\end{equation*}
($f(x,y,z)$是调和函数)当且仅当\begin{equation*}
	\iint_{S}\frac{\partial f}{\partial \vec{n}}\mathrm{d}S=0\quad \left(\forall S\in\Sigma\right)
\end{equation*}
\noindent $\left(2\right)$设$f(x,y,z)$是$\mathbb{R}^3$上的调和函数,$S\in \Sigma$,且$S$围成的有界区域记做$V$。求证:
\begin{equation*}
	f\left(x_{0},y_{0},z_{0}\right)=\frac{1}{4\pi}\iint_{S}\left(f\frac{\cos<\vec{r}\cdot\vec{n}>}{|\vec{r}|^2}+\frac{1}{|\vec{r}|\frac{\partial f}{\partial \vec{n}}}\right)\mathrm{d}S
\end{equation*}
$\left(x_{0},y_{0},z_{0}\right)$是$V$内部的一个定点。$\vec{r}=\left(x-x_{0},y-y_{0},z-z_{0}\right)$。$<\vec{r},\vec{n}>$表示两向量之间的夹角,$|\vec{r}|$表示向量$\vec{r}$的模长。\\
\noindent $\left(3\right)$若函数$f$和闭区间$\left[0,1\right]$上的连续函数$g$在单位球$\Omega=\left\{\left(x,y,z\right)|x^2+y^2+z^2\leq 1\right\}$上满足:\begin{equation*}
	\frac{\partial^2 f}{\partial x^2}+	\frac{\partial^2 f}{\partial y^2}+	\frac{\partial^2 f}{\partial z^2}=g\left(x^2+y^2+z^2\right)
\end{equation*}
求证:
\begin{equation*}
	\iiint_{\Omega}\left(x \frac{\partial f}{\partial x}+y	\frac{\partial f}{\partial y}+z	\frac{\partial f}{\partial z}\right)\mathrm{d}V=\pi\int_{0}^{1}\sqrt{t}\left(1-t\right)g\left(t\right)\mathrm{d}t
\end{equation*}

\subsection*{12}\noindent 已知周期为$2\pi$的函数\begin{equation*}
	f(x)=\frac{1}{4}x\left(2\pi -x\right)\qquad 0\leq x\leq 2\pi
\end{equation*}
\noindent $\left(1\right)$ 通过将$f(x)$展开成傅里叶级数,计算
\begin{equation*}
	\sum_{n=1}^{\infty}\frac{1}{n^2}
\end{equation*}

\noindent$\left(2\right)$计算级数
\begin{equation*}
	\sum_{n=1}^{\infty}\frac{1}{n^4}
\end{equation*}
\subsection*{13}
\noindent 求椭球体
\begin{equation*}
	\frac{x^2}{3}+\frac{y^2}{2}+z^2=1
\end{equation*}
被平面$x+y+z=0$所截得的椭圆面积。
\subsection*{14}\noindent 设$f(x)$当$x\geq 0$时具有连续导数,$f(0)\leq 1$,且满足:\begin{equation*}
	3\left[3+f(x)^2\right]f^{\prime}(x)=2\left[1+f^{2}(x)\right]\exp(-x^2)
\end{equation*}
求证:$f(x)$当$x\geq 0$时有界。
\subsection*{15}\noindent $f(x)$当$x\geq 0$时是有界的连续函数。求证:方程
\begin{equation*}
	y^{\prime\prime}(x)+14y^{\prime}(x)+13y(x)=f(x)
\end{equation*}
的每个解都在$x\geq 0$时有界。
\end{document}
