\documentclass[a4paper,12pt]{article}
\usepackage{ctex}
\usepackage{amsmath,amsbsy,amsfonts}
\usepackage{geometry}
\geometry{left=2.0cm,right=2.0cm,top=2.5cm,bottom=2.5cm}

 
\title{2024年寒假特制试卷:甲卷}
\author{一只波兰立陶宛联邦球}
\date{}

\begin{document}
\maketitle
\section*{注意事项}
本试卷共25题,每题10分,卷面总分150分。考试时间4小时。

范围:高等数学、线性代数、概率论与数理统计。

本试卷中,1、2、15题必做;分别从3/4/5题、6/7/8题、9/10/11题、12/13/14题中每组题选两题作答;从15-20题、21-25题中每组选三题作答。若组内选定的题目数量超过限制,则按题号先后顺序评阅赋分。
\section*{高等数学部分}
\subsection*{1}求以下各极限。
\paragraph{(1)}
\begin{equation*}
	L_{1}=\lim\limits_{x\rightarrow 0}\frac{\sin x^{m}-\sin^{n}x}{x^{n+2}}
\end{equation*}
\paragraph{(2)}
\begin{equation*}
	L_{2}=\lim\limits_{n\rightarrow\infty}\frac{\sum_{i=1}^{n}n^{p}}{n^{p+1}}\quad (n>0)
\end{equation*}
\paragraph{(3)}
\begin{equation*}
	L_{3}=\lim\limits_{x\rightarrow 0}(1-x\sin 2x)^{\frac{1}{x-\ln(1+x)}}
\end{equation*}
\paragraph{(4)}\begin{equation*}
	L_{4}=\lim\limits_{x\rightarrow 0}(e^{-x^2}+\cos x-1)^{\frac{1}{x^2}}
\end{equation*}
\subsection*{2}
假设$f(x)$在$\left[0,1\right]$上二阶可导,且:
\begin{equation*}
	\lim\limits_{x\rightarrow 0^{+}}\frac{f(x)}{x}=2,\int_{0}^{1}f(x)\mathrm{d}x = 1
\end{equation*}
证明以下四个命题:
\paragraph{(1)}存在$0<\xi<1$使得
\begin{equation*}
	f^{\prime}(\xi)=f(\xi)-2\xi+2
\end{equation*}
\paragraph{(2)}存在$0<\eta<1$使得$f^{\prime\prime}\left(\eta\right)=0$
\paragraph{(3)}存在$0<\zeta<1$使得:
\begin{equation*}
	\int_{0}^{\zeta}f(t)\mathrm{d}t+\zeta f(\zeta)=2\zeta
\end{equation*}
\paragraph{(4)}存在$0<\mu<1$使得:
\begin{equation*}
	\mu f(\mu)=2\int_{0}^{\mu}f(t)\mathrm{d}t
\end{equation*}
\subsection*{3}求解以下不定积分:
\paragraph{(1)}
\begin{equation*}
	\int \frac{\mathrm{d}x}{x^4+2x^3+3x^2+2x+1}
\end{equation*}
\paragraph{(2)}
\begin{equation*}
	\int \frac{x\mathrm{d}x}{\sqrt{1+x^2+\sqrt{\left(1+x^2\right)^3}}}
\end{equation*}
\paragraph{(3)}
\begin{equation*}
	\int e^{ax}\cos{bx}\mathrm{d}x\quad(a,b\neq 0)
\end{equation*}
\paragraph{(4)}
\begin{equation*}
	\int \frac{\mathrm{d}x}{(2x+1)^2\sqrt{4x^2+4x+5}}
\end{equation*}
\paragraph{(5)}
\begin{equation*}
	\int \frac{\mathrm{d}x}{(x^2+2)\sqrt{2x^2-2x+5}}
\end{equation*}
\subsection*{4}求解以下反常积分:
\paragraph{(1)}
\begin{equation*}
	I_{1}=\int_{0}^{\infty}\frac{\sin(ax)sin(bx)}{x}\mathrm{d}x\quad a\neq b
\end{equation*}
\paragraph{(2)}
\begin{equation*}
	I_{2}=\int_{0}^{\infty}\frac{\sin^4 (\alpha x)-\sin^4{(\beta x)}}{x}\mathrm{d}x\quad \alpha,\beta >0
\end{equation*}
\paragraph{(3)}
\begin{equation*}
	I_{3}=\int_{0}^{+\infty}e^{-ax}\sin bx\mathrm{d}x\quad(a>0)
\end{equation*}
\paragraph{(4)}
\begin{equation*}
	I_{4}=\int_{+\infty}^{0}x^{n}e^{-x}\mathrm{d}x\quad(n=1,2,\dots)
\end{equation*}
\paragraph{(5)}
\begin{equation*}
	I_{5}=\int_{-\pi}^{\pi}\frac{1-r^{2}}{1-2r\cos x +r^2}\quad(0<r<1)
\end{equation*}
\subsection*{5}
\paragraph{(1)}
对于任意$n\geq 2$,证明:
\begin{equation*}
	\int_{\pi}^{n*\pi}\frac{\left|\sin x\right|}{x}\mathrm{d}x>\frac{2}{\pi}\ln\frac{n+1}{2}
\end{equation*}
\paragraph{(2)}假设$f(x)$在$\left[0,\pi\right]$上连续,且:
\begin{equation*}
	\int_{0}^{\pi}f(x)\sin x\mathrm{d}x=\int_{0}^{\pi}f(x)\cos x\mathrm{d}x
\end{equation*}
请说明$f(x)$在$\left[0,\pi\right]$上至少有两个零点。
\subsection*{6}\noindent 假设$u=f\left(x,y\right)$具有二阶连续偏导数,满足:
\begin{equation*}
	4\frac{\partial^2 u}{\partial x^2}+12\frac{\partial^2 u}{\partial x\partial y}+5\frac{\partial^2 u}{\partial y^2}=0
\end{equation*}
请使用合适的线性替换,将方程化为:
\begin{equation*}
	\frac{\partial ^2 u}{\partial \xi \partial \eta}=0
\end{equation*}
其中,$\xi = x+ay,\eta =x+by$。
\subsection*{7}\noindent
已知函数$z=z(x,y)$由
\begin{equation*}
	\left(x^2+y^2\right)z+\ln z+2(x+y+1)=0
\end{equation*}确定。求其极值。
\subsection*{8}\noindent 求函数$f(x,y)=x+y+xy$在曲线$C:x^2+y^2+xy=3$上的最大方向导数。
\subsection*{9}\noindent
假设$f(x,y)$在单位圆域内具有连续的偏导数,且在单位圆圆周上取值为零。\\若$D:=\left\{(x,y)|\varepsilon^2\leq x^2+y^2\leq 1\right\}$,求证:
\begin{equation*}
	f(0,0)=\lim\limits_{\varepsilon\rightarrow 0^{+}}(-\frac{1}{2\pi})\iint_{D}\frac{x\frac{\partial f}{\partial x}+y\frac{\partial f}{\partial y}}{x^2+y^2}\mathrm{d}x\mathrm{d}y
\end{equation*}
\subsection*{10}
\noindent$\Omega$是曲面$(x^2+y^2+z^2)^2=2xy$与坐标平面围成的区域在第一卦限内的部分。试计算:
\begin{equation*}
	\iiint_{\Omega}\frac{xyz}{x^2+y^2}\mathrm{d}x\mathrm{d}y\mathrm{d}z
\end{equation*}
\subsection*{11}\noindent
假设$f(x,y)$在区域$D:=\left\{(x,y)|x^2+y^2\leq a^2\right\}$上具有一阶连续偏导数,且满足:
\begin{equation*}
	f(x,y)|_{x^2+y^2=a^2}=a^2\quad \max\limits_{(x,y\in D)}\left[\left(\frac{\partial f}{\partial x}\right)^2+\left(\frac{\partial f}{\partial y}\right)^2\right]=a^2\quad(a>0)
\end{equation*}
求证:
\begin{equation*}
	\left|\iint_{D}f(x,y)\mathrm{d}x\mathrm{d}y\right|\leq\frac{4}{3}\pi a^4
\end{equation*}
\subsection*{12}\noindent
假设$f(x)$在$\mathbb{R}$上可导,且$f(x)=f(x+2)=f(x+\sqrt{3})$。请使用傅里叶级数理论论证该函数为常数。

\subsection*{13}
\noindent 求幂级数
\begin{equation*}
	\sum_{n=1}^{\infty}\frac{x^{n-1}}{n\times 2^n}
\end{equation*}
的收敛域与和函数,并计算$\sum_{n=1}^{\infty}\frac{1}{n\times2^n}$的值。

\subsection*{14}
\noindent 数列
$\left\{a_{n}\right\}(n\geq 0)$由初值条件$a_{0}=1,a_{1}=2$与递推公式\begin{equation*}
	a_{n+2}=\frac{2}{n+2}a_{n+1}-\frac{1}{(n+2)(n+1)}a_{n}
\end{equation*}
决定。试求出其通项公式。

\subsection*{15}
\noindent 假设$f\left(x\right)$可导,且\begin{equation*}
	x=\int_{0}^{x}f\left(t\right)\mathrm{d}t+\int_{0}^{x}tf\left(t-x\right)\mathrm{d}t
\end{equation*}
试求:
\begin{equation*}
	\int_{-\frac{\pi}{4}}^{\frac{3\pi}{4}}\left|f\left(x\right)\right|^6\mathrm{d}x
\end{equation*}
\section*{线性代数部分}
\subsection*{16}
\begin{equation*}
	A = \begin{pmatrix}
	3&3&2\\2&3&2\\2&2&3\\	
	\end{pmatrix}\quad
P= \begin{pmatrix}
	0&1&0\\1&0&1\\0&0&1\\	
\end{pmatrix}\quad 
B=P^{-1}A^{*}P
\end{equation*}
$E$为三阶单位矩阵,$A^{*}$为$A$的伴随矩阵。求$B+2E$的特征值与特征向量。
\subsection*{17}
\noindent 已知秩为2的二次型
\begin{equation*}
	f(x_{1},x_{2},x_{3})=\left(1-a\right)x_{1}^2+\left(1-a\right)x_{2}^2+2x_{3}^2+2(1+a)x_{1}x_{2}
\end{equation*}
请求出正交变换将其化为标准型。
\subsection*{18}
\noindent 设$A$为三阶矩阵,$\alpha_{1},\alpha_{2},\alpha_{3}$是线性无关的三维列向量。且满足:
\begin{equation*}
	\begin{aligned}
		A\alpha_{1}&=\alpha_{1}+\alpha_{2}+\alpha_{3}\\
		A\alpha_{2}&=2\alpha_{2}+\alpha_{3}\\
		A\alpha_{3}&=2\alpha_{2}+3\alpha_{3}
	\end{aligned}
\end{equation*}
\paragraph{(1)}求矩阵$B$满足$A\left(\alpha_{1},\alpha_{2},\alpha_{3}\right)=\left(\alpha_{1},\alpha_{2},\alpha_{3}\right)B$。
\paragraph{(2)}求矩阵$B$的特征值。
\paragraph{(3)}求可逆矩阵$P$满足$P^{-1}AP$为对角阵。
\subsection*{19}
\noindent 考虑$n$阶方阵$A$和$B$及$n$阶单位矩阵$E$,$rank(A)+rank(B)<n$。同时$AB=A+B$。证明:
\paragraph{(1)}$A-E$可逆并求出其逆矩阵。
\paragraph{(2)}$AB=BA$
\paragraph{(3)}$A$和$B$的特征向量完全相同。

\subsection*{20}\noindent 完成以下证明。
\paragraph{(1)}假设$A$是$m\times n$的实矩阵。$E$是单位矩阵。$B=\lambda E+A^{T}A$。证明:当$\lambda>0$时,B是正定矩阵。
\paragraph{(2)}假设$A$为$m$阶正定实矩阵,$B$为$m\times n$阶实矩阵。证明:$B^{T}AB$为正定矩阵的充分必要条件是$rank(B)=n$。
\section*{概率论与数理统计部分}
\subsection*{21}\noindent 假设以下描述的事件$A,B,C$均有意义。完成以下证明。
\paragraph{(1)}如果$P(A)>0$,有$P(AB|A)>P(AB|A\cup B)$。
\paragraph{(2)}若$P(A|B)=1$,则$P(\bar{B}|\bar{A})=1$。
\paragraph{(3)}若$P(A|C)\geq P(B|C)$,$P(A|\bar{C})\geq P(B|\bar{C})$,则$P(A)\geq P(B)$。
\subsection*{22}\noindent 设随机变量$X$服从参数0.5的指数分布。求证:$Y=1-\exp(-2X)$服从$\left(0,1\right)$上的均匀分布。
\subsection*{23}\noindent 设随机变量$X_{1}$,$X_{2}$,$X_{3}$互相独立,其中$X_{1}$和$X_{2}$服从标准正态分布。且$P{X_{3}=0}=P{X_{3}=1}=0.5$。定义$Y=X_{3}X_{1}+\left(1-X_{3}\right)X_{2}$。
\paragraph{(1)}用标准正态分布函数表示二维随机变量$\left(X_{1},Y\right)$的分布函数。
\paragraph{(2)}求证:$Y$服从标准正态分布。
\subsection*{24}\noindent 某种导线,要求其电阻的标准差不得超过0.005欧姆,在生产的一批产品中取9根,测得样本标准差为0.007欧姆。假设总体服从参数未知的正态分布。在显著性水平$\alpha=0.05$下,能否认为这批导线的电阻标准差显著地比平均值偏大?
\subsection*{25}\noindent 已知随机变量$Y$的密度函数和给定$Y=y$条件下随机变量$X$的密度函数分别为:
\begin{equation*}
	p_{Y}(y)=\left\{\
	\begin{aligned}
		5y^4&\quad 0<y<1\\0&\quad \mbox{其他}
	\end{aligned}
\right.
\end{equation*}
\begin{equation*}
	p(x|y)=\left\{
	\begin{aligned}
		\frac{3x^2}{y^3}\quad&0<x<y<1\\0\quad&\mbox{其他}
	\end{aligned}
	\right.
\end{equation*}
求概率$P(X>0.5)$。
\end{document}
